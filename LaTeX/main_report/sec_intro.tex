\sectionCenteredToc{ВВЕДЕНИЕ}
\label{sec:intro}

Для каждого центра обработки данных (ЦОД) научно-исследовательского института необходимо иметь грамотно проработанную и качественно настроенную локальную компьютерную сеть (ЛКС). В задачи ЦОД входит сбор и обработка больших объемов информации, вычисления, статистический анализ и визуализация результатов исследований, хранение научных данных и результатов исследований. Исходя из этого создание локальной компьютерной сети является важной задачей, без нее невозможно представить слаженную работу всех участников центра, своевременное получение точных данных, а также быструю и качественную работу всех участников проекта. 

Целью данного курсового проекта является разработка ЛКС для ЦОД научно-исследовательского института. В рамках проекта выдвинуто требование построить сеть, которая должна поддерживать бесперебойную работу в случае большого скопления людей в центре, высокоскоростную линию беспроводной сети и, помимо наличия стационарных точек подключения к сети, так же иметь в распоряжении блейд-сервер, принтеры и интерактивные доски для обеспечения участников всем необходимым оборудованием для комфортной работы.

Центр владеет одним этажом в здании, а количество пользовательских станций, которые требуется расположить на этаже, равно трем. Исходя из этого можно располагать ПК достаточно удаленно друг от друга, давая больше свободного пространства для каждого из участников проекта.

Отдельно стоит выделить, что обеспечение безопасности данных и защита от утечек информации является важной задачей в проекте. Для ЦОД недопустимо хищение и дальнейшее распространение 
научных исследований, следовательно надо уделить особое внимание наличию средств защиты.

Недостатком является выбор помещения: здание имеет своеобразную форму и небольшую площадь на этаже, что при условии увеличении количества участников центра может вызвать проблемы с размещением.

В соответствии с поставленной целью были определены следующие задачи:
\begin{enumerate_num}
    \item Изучить материалы и требования для реализации курсового проекта.
    \item Разработать структуру сети и структурную схему.
    \item Определиться с выбором устройств и обосновать их выбор.
    \item Составить функциональную схему.
    \item Сделать выводы и написать руководство пользователя.
\end{enumerate_num}