\section{ПРОЕКТИРОВАНИЕ СТРУКТУРИРОВАННОЙ КАБЕЛЬНОЙ СИСТЕМЫ}
\label{sec:ckc}

\subsection{Общая информация о здании}

Здание, в котором проектируется ЛКС имеет п-образную форму. Общая площадь помещений составляет 30 квадратных метров и находится на втором этаже. Этаж разделен на пять помещений: первая комната для разработчика (№201 площадью 7.57 квадратных метров), вторая комната разработчиков (№202 площадью 9.75 квадратных метров), серверная комната (№203 площадью 5.6 квадратных метров), и лестничная клетка (№204 площадью 6 квадратных метров), и санузел (№205 площадью 0.88~квадратных метров). Внешние стены здания выполнены из газобетонных блоков толщиной 200~мм, стены внутри помещения выполнены из шлакоблока толщиной 100~мм. Под потолком на расстоянии 20~см от потолка и 5~см от стены, находится кабельный лоток размером 50х100~мм, с перегородкой для разделения силовых и сетевых кабелей. План здания представлен в приложении В.

\subsection{Распределительные пункты}

Для выбора телекоммуникационного шкафа необходимо было в первую очередь учитывать размеры блейд-сервера. Глубина шкафа должна быть более 820~мм, а ширина и высота более 450~мм, так же в шкаф необходимо установить маршрутизатор, поэтому высота должна браться с запасом. По характеристикам было выбрано два шкафа: TWT~19″ серии Business TWT-CBB-42U-6×10-P1~\cite{business_skaff} и 47U~ЦМО ШТК-М-47.6.10-44АА-9005~\cite{cmo_skaff}. Оба шкафа подходят для ЛКС, они оба имеют разные варианты дверей, места для установки вентиляционных модулей, имеют винтовые опоры позволяющие компенсировать неровности пола. В итоге был выбран шкаф от производителя ЦМО, по причине лучшей стоимости в 2566 белорусских рублей~\cite{cmo_skaff}.

\subsection{Изделия и материалы}

Для монтажа кабелей из лотка к информационным розеткам и к беспроводным точкам доступа используются кабель-каналы (коробы). Размеры короба выбираются с учетом диаметра кабеля UTP категории 5e и его возможного увеличения, а также учитывается, что кабели должны занимать 50\% площади поперечного сечения короба. Исходя из номинального диаметра кабеля UTP категории 5e, равного 0.51~мм, и увеличения диаметра на 10\%, подходящий размер поперечного сечения короба составляет 15х10~мм.

Информационные розетки бывают разных видов, для разных категорий кабелей, в нашей сети используется витая пара категории 5e поэтому розетки будем выбирать для данной категории, так же по типу установки необходимо рассматривать только внешние, потому что у нас нет возможности сделать отверстия в стенах и кабель мы спускаем по небольшому коробу. Для Legrand Mosaic 76564~\cite{mosaic76564_legrand} и RJ45 SCHNEIDER ELECTRIC Glossa GSL000181K~\cite{glossagsl000181k_schneider_electric}. Обе розетки имеют стильный дизайн и подходят по характеристикам, однако розетка от компании SCHNEIDER обладает более прочным металлическим основанием и так же является более бюджетным вариантом. Стоимость: 18 белорусских рублей~\cite{glossagsl000181k_schneider_electric}.

\subsection{Выбор среды передачи данных}

Исходя из того что в помещениях отсутствуют сильные электромагнитные помехи, то может использоваться кабель вида UTP. С учетом выбранного активного сетевого оборудования для монтажа достаточно было использовать не экранированную витую пару категории 5e~\cite{vitaya_para}. В компании, занимающейся обработкой данных, как правило, нет сильных источников помех или особо чувствительного оборудования, поэтому экранирование витой пары применяться не будет. Основные характеристики витой пары: максимальная длинна в 100~метров, максимальная скорость до 1~Гбит/с, поддержка Power over Ethernet. Может применяться для передачи видеосигналов и телефонной связи. 

В процессе выбора пассивного оборудования, были выбраны: витая пара Legrand UTP категории 5e PVC 4~\cite{kabel-legrand}. Так же при выборе витой пары сравниваливсь кабеля от производителя SkyNet~\cite{kabel-skynet} и Corex~\cite{kabel-corex}. Все они являются неэкранированными и по характеристикам одинаковы. Поэтому было решено взять кабель средней ценовой категории. В качестве оптоволокна был выбран патч-корд оптический SC/APC~-- SC/APC 9/125~sm~1м LSZH~\cite{opticheskiy-simplex-sc-apc-sc-apc}. Этот кабель обеспечивает хорошую производительность и надежность и позволяет передавать информацию на большой скорости, что крайне важно для ЦОД. Так же был выбран трансивер оптический GateRay GR-S01-W5540S SFP~\cite{transivery_opticheskie_gr_s1_w5540s}, для соединения оптоволокна с SFP портом.

\subsection{Расчет качества связи беспроводной сети}

Для проведения расчета покрытия беспроводной сетью всех помещений на этаже ЦОД научно-исследовательского института и определения достаточности выбранной точки беспроводного доступа Audience LTE6 kit, требуется учесть несколько ключевых параметров таких как площадь помещения, характеристики стен и перегородок.

Беспроводная сеть должна покрывать всю площадь этажа и обеспечивать до 50 соединений. Внешние стены здания и стены этажа состоят из газобетонных, внутренние стены, в свою очередь, выполнены из шлакоблока. Высота этажа составляет 2.7~м.

Для расчета затухания радиоволн в беспрепятственной воздушной среде используется упрощенная формула:

\begin{equation}
  L = 32.44 + 20 \cdot \lg{F} + 20 \cdot \lg{D}, \textup{~дБ}
\end{equation}
\begin{explanationx}
  \item[где] $F$ -- частота сигнала в ГГц;
  \item  $D$ – расстояние в метрах от точки доступа.
\end{explanationx}

Чувствительность устройств обычно находится в пределах от -65 до -75~дБ. Форма здания п-образная, предположим размещение точки доступа на стене в 30~см от потолка. Высота потолка – 2,5~м, максимальная длина до наиболее удалённых внешних стен от центра~-- 4~м и 4,3~м. Рассчитаем расстояние до наиболее удаленной точки помещения (левого нижнего угла):

\begin{equation}
  D = \sqrt{l^2 + w^2 + h^2} = \sqrt{4^2 + 4.3^2 + 2.2^2} = 6.27 \textup{~м},
\end{equation}
\begin{explanationx}
  \item[где] $l$ -- длина;
  \item  $w$ -- ширина;
  \item  $h$ -- высота.
\end{explanationx}

Рассчитаем затухание радиоволн $L_{2.4}$ для частоты 2.4~ГГц и $L_5$ для частоты 5~ГГц:
\begin{gather}
  L_{2.4} = 32.44 + 20 \cdot \lg{2.4} + 20 \cdot \lg{6.27} = 55.99 \textup{~дБ}, \\
  L_{5} = 32.44 + 20 \cdot \lg{5} + 20 \cdot \lg{6.27} = 62.36 \textup{~дБ}.
\end{gather}

Необходимо учесть затухание на конструкционных элементах здания. Затухание радиоволны при прохождении газобетонной стены толщиной 200~мм составляем 8~дБ, при прохождении шлакоблока толщиной 100~мм~-- 4~дБ~\cite{zatux}. Учитывая, что несущие стены состоят из газобетонных материалов и максимальное количество препятствующих стен равно двум, а количество внутренних стен состоящих из шлакоблока равно одной. То затухание радиоволн при прохождении стен $L_\textup{макс.конст.} = 2 \cdot L_\textup{газобетона} + L_\textup{шлакоблока} = 2 \cdot 8 + 4 = 20 \textup{~дБ}$.

Также стоит учесть возможное затухание за счет взаимного размещения оборудования $L_\textup{обор} = 5 \textup{~дБ}$.

Тогда максимальное затухание сигнала в помещениях организации составляет:
\begin{gather}
  L_{{2.4}_\textup{макс.}} = L_\textup{макс. конст.} + L_\textup{2,4} + L_\textup{обор.} = 20 + 55.99 + 5 = 80.99 \textup{~дБ},\\
  L_{{5}_\textup{макс.}} = L_\textup{макс. конст.} + L_\textup{5} + L_\textup{обор.} = 20 + 62.36 + 5 = 87.36 \textup{~дБ}.
\end{gather}

Тогда минимальная мощность сигнала в помещении будет равна:
\begin{gather}
  S_{2.4} = S_\textup{маршр.} + L_{{2,4}_\textup{макс.}} = 26 - 80.99 = -54.99 \textup{~дБ},\\
  S_{5} = S_\textup{маршр.} + L_{{5}_\textup{макс.}} = 28 - 87.36 = -59.36 \textup{~дБ}.
\end{gather}

Такой уровень сигнала является достаточным для покрытия точкой беспроводного доступа всего этажа. А наличие трёх частот обеспечит отказоустойчивость в случае большого скопления людей.

\subsection{Размещение и монтаж оборудования}

В комнате разработчика (помещение №201) возле информативной розетки размещается компьютер (системный блок Jet Wizard 5i9400FD, монитор ASUS VA24DQ, клавиатура Microsoft Wired Keyboard 600, мышь Dell Optical Mouse MS116) и интерактивная доска SMART Board~MX265-V2, которая устанавливается на расстоянии 150~см от пола и не менее 30~см от ближайшей стенки. Так же в середине комнаты, на потолке размещается беспроводная точка доступа Mikrotik Audience LTE6 kit. 

От маршрутизатора в помещение №201 протянуто два кабеля витой пары UTP категории 5e в кабель-лотке над потолком. Первый кабель соединяется с информационной розеткой RJ45 SCHNEIDER ELECTRIC Glossa, которая в свою очередь используется для подключения компьютера пользователя. Второй кабель соединяется с информационной розеткой которая необходима для подключения интерактивной доски.

В помещении №202 расположено два компьютера (системный блок Jet Wizard 5i9400FD, монитор ASUS VA24DQ, клавиатура Microsoft Wired Keyboard 600, мышь Dell Optical Mouse MS116), каждый из которых соединен с принтером HP OfficeJet Pro 8210 посредством кабеля USB который идет к комплекте с ним. 

От маршрутизатора в помещение №202 протянуто четыре кабеля витой пары UTP категории 5e в кабель-лотке над потолком. Три кабеля соединяется с информационными розетками RJ45 SCHNEIDER ELECTRIC Glossa, которые в свою очередь используется для подключения компьютеров пользователей. Четвертый идет над потолком к беспроводной точке доступа.

В серверной комнате (помещение №203) установлен напольный телекоммуникационный шкаф ЦМО ШТК-М-47.6.10-44АА-9005 на растоянии 50~см от дальней стены, посередине. Это необходимо потому что в телекоммуникационном шкафу расположен маршрутизатор CCR2004-16G-2S+PC и блейд-сервер HP Blade System C7000 G3 и к серверу необходим доступ с разных сторон. К блейд-серверу из маршрутизатора идет два кабеля: первый оптоволоконный, второй витая пара категории 5e, для администрирования.