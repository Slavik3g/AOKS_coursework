\section{ФУНКЦИОНАЛЬНОЕ ПРОЕКТИРОВАНИЕ}
\label{sec:func}

В данном разделе пояснительной записки описывается и проводится функциональное проектирование заданной локальной компьютерной сети. Здесь дается более подробное описание функционирования программной и аппаратной составляющих разрабатываемой сети, а именно: обоснование выбора и характеристики используемого оборудования, приведена схема адресации заданных устройств, а также итоговая функциональная схема. 

Функциональная схема представлена в приложении Б.

\subsection{Обоснование выбора активного сетевого оборудования}

% Одним из пожеланий заказчика является возможность просмотра видео посредством беспроводной сети и работоспособность беспроводной сети при скоплении людей
% Исходя из вышесказанного было принято использовать следующее активное сетевое оборудование:
% \begin{enumerate_num}
%     \item Марштуризатор MikroTik RB4011iGS+RM.
%     \item Коммутатор MikroTik CSS326-24G-2S
%     \item Беспроводная точка доступа MikroTik RBcAP AC (RBcAP-AC-5n)
% \end{enumerate_num}

\subsubsection{Обоснование выбора маршрутизатора}

Элементами разрабатываемой локальной компьютерной сети являются три пользовательские станции, блейд-сервер и две точки доступа. Принтеры и интерактивные доски подключаются непосредственно к станциям и не являются частью общей сети. Таким образом, необходимости в коммутаторе нет. Все устройства можно подключить к маршрутизатору, который также обеспечит выход в интернет.

Главными требованием к маршрутизатору является наличие двух портов SFP для возможности подключения блейд-сервера по оптоволокну, для большей пропускной способности и наличие оптимального количества портов Gigabit Ethernet. Так же требование заказчика было чтоб оборудование было от компании MikroTik.

Для рассмотрения выше были выбраны следующие флагманские модели: MikroTik L009UiGS-RM~\cite{L009UiGS-RM}, MikroTik CCR2116-12G-4S+~\cite{CCR2116-12G-4S}, MikroTik CCR2004-16G-2S+PC~\cite{CCR2004-16G-2SPC}. L009UiGS-RM подходила по мощности, но не было запаса в количестве портов Gigabit Ethernet для возможного будущего расширения сети, модель CCR2116-12G-4S+ по количеству SFP+ портов была избыточна для данной сети. Маршрутизатор CCR2004-16G-2S+PC обладает необходимыми характеристиками, так что он в полной мере подходит для данной ЛКС.

Маршрутизатор был выбран, потому что он является мощным маршрутизатором со встроенным пассивным охлаждением. Главным критерием было наличие 18 проводных портов, включая 16 Gigabit Ethernet портов и два порта SFP+ (10~Гбит/с). Каждая группа из восьми Gigabit Ethernet портов подключена к отдельному чипу коммутации семейства Marvell Amethyst. MikroTik CCR2004-16G-2S+PC обладает следующими характеристиками:

\begin{itemize}
    \item процессор: AL32400 1,2~ГГц, 4~ядра, архитектура ARM (64~бит);
    \item ОЗУ: 4~ГБ DDR4;
    \item ПЗУ: 128~MБ NAND;
    \item сетевой интерфейс: 16x Ethernet (1~Гбит/с), 2x SFP+ (10~Гбит/с);
    \item последовательный консольный порт: 1xRJ45;
    \item размеры: 272x195x44~мм;
    \item охлаждение: пассивное;
    \item операционная система: RouterOS;
    \item дополнительно: датчик температуры процессора;
    \item стоимость: 4110 белорусских рублей~\cite{CCR2004-16G-2SPC}.
\end{itemize}

% \begin{table}[ht]
%     \caption{Характеристики CCR2004-16G-2S+PC}
%     \label{table:func:comutator}
%     \begin{tabular}{| >{\raggedright}m{0.50\textwidth}
%                     | >{\raggedright\arraybackslash}m{0.50\textwidth}|}
%         \hline
%         \centering Параметр & \centering\arraybackslash Значение \\
%         \hline
%         Процессор & AL32400 1,2 ГГц, 4 ядра, архитектура ARM (64 бит) \\
%         \hline
%         ОЗУ & 4 ГБ DDR4\\
%         \hline
%         ПЗУ & 128 MБ NAND \\
%         \hline
%         Сетевой интерфейс & 16x Ethernet (1 Гбит/с), 2x SFP+ (10 Гбит/с) \\
%         \hline
%         Последовательный консольный порт & 1x RJ45 \\
%         \hline
%         Число гигабитных портов в коммутаторе (2 чипа коммутации) & 8 \\
%         \hline
%         Размеры & 272 x 195 x 44 мм \\
%         \hline
%         Охлаждение & Пассивное \\
%         \hline
%         Операционная система & RouterOS (версия 7) Level 6 \\
%         \hline
%         Дополнительно & Датчик температуры процессора \\
%         \hline
%     \end{tabular}
% \end{table}

% \subsubsection{Коммутатор}

% После изучения рынка были выбраны два коммутатора MikroTik CSS326-24G-2S и MikroTik CRS518-16XS-2XQ-RM. Оба коммутатора имеют схожие характеристики и поддерживают основные функции, необходимые для коммутации и маршрутизации. Однако между ними есть и некоторые существенные различия.

% \begin{itemize}
%     \item MikroTik CSS326-24G-2S имеет больше портов Ethernet, чем MikroTik CRS518-16XS-2XQ-RM. Это делает его более подходящим для сетей с большим количеством устройств;
%     \item MikroTik CRS518-16XS-2XQ-RM поддерживает порты XFP, которые позволяют подключать устройства с высокой пропускной способностью, например, оптические трансиверы;
%     \item MikroTik CRS518-16XS-2XQ-RM также поддерживает аппаратное ускорение маршрутизации, что может значительно улучшить производительность коммутатора, особенно если вы планируете использовать его в больших сетях;
%     \item MikroTik CRS518-16XS-2XQ-RM является более дорогим вариантом, чем MikroTik CSS326-24G-2S;
%     \item MikroTik CSS326-24G-2S - это коммутатор Layer 2 с 24 портами Gigabit Ethernet и двумя портами SFP+. Он предназначен для использования в небольших и средних сетях, где требуется высокая производительность и надежность.
% \end{itemize}

% Исходя из всех различий было принято решение взять MikroTik CSS326-24G-2S потому что он более доступен и имеет достаточное количество портов для устройств и возможности будущего масштабирования. Он является хорошим выбором для ЦОС, так как он обеспечивает высокую производительность, надежность и гибкость.

% Коммутатор имеет следующие характеристики:
% \begin{itemize}
%     \item 24 порта Gigabit Ethernet с поддержкой автосогласования скорости и дуплекса;
%     \item 2 порта SFP+ со скоростью передачи данных до 10 Гбит/с;
%     \item Поддержка Layer 2 коммутации;
%     \item Поддержка протоколов VLAN, STP, RSTP и MSTP;
%     \item Поддержка QoS (приоритизации трафика).
% \end{itemize}

\subsubsection{Обоснование выбора точки беспроводного доступа}

Для увеличения плотности сети было принято решение рассматривать двух или трех диапазонные точки доступа, потому что они обеспечат поддержку наибольшего количество устройств благодаря использованию разных частотных диапазонов.

Для рассмотрения были выбраны следующие модели: MikroTik Audience LTE6 kit~\cite{audiencelte6kit}, MikroTik wAP R RBwAPR-2nD~\cite{rbwapr2nd}, MikroTik LHG LTE18 kit~\cite{lhglte18kit}. Модель wAP R является двухдиапазонным беспроводным маршрутизатором уступает Audience LTE6 kit и LHG LTE18 kit в скорости передачи, LHG LTE18 kit в том же ценовом диапазоне, что и Audience LTE6 kit, но имеет ориентированность на применение вне помещений и является менее выгодным предложением для использовании в помещениях.

В наибольшей степени требованиям отвечает точка доступа Mikrotik Audience LTE6 kit. Это трехдиапазонная (одна 2,4~ГГц и две 5~ГГц) точка доступа с поддержкой LTE и технологией построения сетей. Также на ней есть два Gigabit Ethernet порта, которые объединены через встроенный свитч-чип. На устройстве отсутствуют SFP- и USB-интерфейсы и разъем под карту памяти. Отличительной особенностью данной модели является то, что устройство имеет стильный дизайн и имеет два отдельных Wi-Fi-модуля для работы на частоте 5~ГГц. 

Технические характеристики Audience LTE6 kit:

\begin{itemize}
    \item операционная система: RouterOS;
    \item Ethernet-интерфейсов: 2;
    \item архитектура: ARM 32~бита;
    \item процессор: IPQ-4019;
    \item стандарты WI-FI: 802.11a/b/g/n/ac;
    \item размеры: 100x235x97~мм;
    \item допустимая температура окружающей среды: от -30 до +60 °C;
    \item ожидаемое время работы до отказа: 200 000 часов при 25 °C;
    \item частота процессора: 716~МГц;
    \item стоимость: 1411 белорусских рублей~\cite{audiencelte6kit}.
\end{itemize}

Основные характеристики передачи данных на разных диапазонах приведены в таблице~\ref{table:func:speed_diapazon}. Приведенные в таблице~\ref{table:func:speed_diapazon} значения для физического уровня модели OSI. В реальной жизни даже в самых идеальных условиях для передачи полезных данных получится достичь не более 50\% от указанного значения~\cite{real_speed}.

\begin{table}[ht]
    \caption{Основные характеристики передачи данных на разных диапазонах точки доступа Audience LTE6 kit}
    \label{table:func:speed_diapazon}
    \begin{tabular}{|>{\raggedright}m{0.15\textwidth}|>{\raggedright}m{0.52\textwidth}|>{\raggedright\arraybackslash}m{0.25\textwidth}|}
        \hline
        \centering Частота & \centering Параметр & \centering\arraybackslash Значение \\
        \hline
        2,4~ГГц & Максимальная скорость & 300~Мбит/с \\
        \cline{2-3}
                 & Модель чипа & IPQ-4019 \\
        \cline{2-3}
                 & Максимальная мощность передатчика & 26~дБм (800~мВт) \\
        \cline{2-3}
                 & Количество радиоканалов & 2 \\
        \hline
        5 ГГц & Максимальная скорость & 867~Мбит/с \\
        \cline{2-3}
                 & Модель чипа & IPQ-4019 \\
        \cline{2-3}
                 & Максимальная мощность передатчика & 28~дБм (630~мВт) \\
        \cline{2-3}
                 & Количество радиоканалов & 2 \\
        \hline
        5 ГГц & Максимальная скорость & 1733~Мбит/с \\
        \cline{2-3}
                 & Модель чипа & QCA9984 \\
        \cline{2-3}
                 & Максимальная мощность передатчика & 28~дБм (1584~мВт) \\
        \cline{2-3}
                 & Количество радиоканалов & 4 \\
        \hline
    \end{tabular}
\end{table}

\subsection{Обоснование выбора серверного оборудования}

\subsubsection{Обоснование выбора блейд-сервера}

Блейд-сервер~-- компьютерный сервер с компонентами, вынесенными и обобщенными в корзине для уменьшения занимаемого пространства. Блейд-система обязательно должна состоять из блейд-шасси и блейд-сервера.

Для сравнения были выбраны три блейд-шасси Dell PowerEdge MX7000~\cite{poweredge-mx7000-dell}, HP Blade System C7000 G3~\cite{shassi-hp-c7000-g3-enclosure}, IBM BLADECENTER H CHASSIS~\cite{bladecenter_chassis_h}. Все системы предоставляют хорошую производительность, однако системы от компании Dell и IBM являются самыми дорогими вариантами, а по характеристикам практически не уступают блейд-шасси от компании HP. Поэтому в качестве шасси был выбран BladeSystem c7000 G3 от компании HP. Он полностью соответствует необходимыми техническими характеристиками, поддерживает централизованное управление, которое упрощает настройку и обслуживание системы, имеет возможность масштабирования, путем добавления дополнительных шасси, это позволяет системе легко расширяться по мере роста потребностей бизнеса, а так же он хорошо задокументирован официальными производителями. Его технические характеристики:

\begin{itemize}
    \item форм-фактор: 10U;
    \item отсеки блейд-серверов: 16 отсеков для устройств половинной высоты или 8 отсеков для устройств полной высоты;
    \item отсеки коммуникационных модулей: восемь межблочных отсеков для коммуникационных модулей (Ethernet, Fibre Channel, Infiniband, SAS);
    \item электропитание: до шести блоков питания с горячей заменой при мощности 2250~Вт;
    \item охлаждение: 10 вентиляторов Active Cool с возможностью горячей замены;
    \item размеры: 447x442x813~мм;
    \item стоимость: 5125 белорусских рублей~\cite{shassi-hp-c7000-g3-enclosure}.
\end{itemize}

% \begin{table}[ht]
%     \caption{Характеристика BladeSystem c7000 G3}
%     \label{table:func:blade-server}
%     \begin{tabular}{| >{\raggedright}m{0.397\textwidth}
%                     | >{\raggedright\arraybackslash}m{0.55\textwidth}|}
%         \hline
%         \centering Параметры & \centering\arraybackslash Значение\\

%         \hline
%         Блейд-шасси & HP BladeSystem c7000 Enclosure (507016-B21)  \\
        
%         \hline
%         Отсеки блейд-модулей & 16 отсеков для устройств половинной высоты или 8 отсеков для устройств полной высоты \\
        
%         \hline
%         Отсеки коммуникационных модулей & 8 межблочных отсеков для коммуникационных модулей (Ethernet, Fibre Channel, Infiniband, SAS) \\

%         \hline
%         Электропитание &  До 6 блоков питания с горячей заменой при мощности 2250 Вт \\
        
%         \hline
%         Охлаждение & 10 вентиляторов Active Cool с возможностью горячей замены \\

%         \hline
%         Форм-фактор & 10U \\

%         \hline
%         Размеры  & Высота: 447мм, ширина: 442 мм, толщина: 813мм \\

%         \hline
%     \end{tabular}
% \end{table}

Для выбранного блейд-шасси нужно подобрать сами сервера. На данный момент у компании HP есть три поколения HPE ProLiant BL460c Gen8~\cite{bl460c-gen8}, HPE ProLiant BL460c Gen9~\cite{bl460c-gen9} и HPE ProLiant BL460c Gen10~\cite{bl460c-gen10}. Основные отличия в годе выпуска, и соответственно в характеристиках, у более старого поколения они будут ниже чем у флагманов. Из-за невозможности покупки самого последнего поколения и очень завышенных цен, было принянято решение что для ЦОС будет достаточно предыдущего поколения. Пэтому в качестве комплектации будет восемь лезвий HP Proliant BL460c G9 727031-B21. Основные характеристики блейд-лезвия HP Proliant BL460c G9 727031-B21:

\begin{itemize}
    \item процессор: два процессора Intel Xeon E5-2600 v3;
    \item ОЗУ: 16 слотов памяти;
    \item дисковый контроллер: Smart Array P244br RAID контроллер с модулем энергонезависимой флеш-памяти (FBWC) размером 1~Гб. Поддержка RAID 0,1;
    \item поддержка внутренних дисков: два отсека для установки накопителей SSD;
    \item сетевой контроллер: два интегрированных двух портовых контроллера NC551i FlexFabric 10~Гб;
    \item графическая подсистема: ATI Radeon 50 с 16~Мб памяти;
    \item стоимость: 1984 белорусских рублей~\cite{bl460c-gen9}.
\end{itemize}

% \begin{table}[ht]
%     \caption{Характеристика HP Proliant BL685c G7}
%     \label{table:func:blade-server-blades}
%     \begin{tabular}{| >{\raggedright}m{0.397\textwidth}
%                     | >{\raggedright\arraybackslash}m{0.55\textwidth}|}
%         \hline
%         \centering Параметры & \centering\arraybackslash Значение\\

%         \hline
%         Процессор & 4 процессора AMD Opteron 6100 серии  \\
        
%         \hline
%         Чипсет & AMD SR5690 и SP5100 \\
        
%         \hline
%         Оперативная память & 32 слота памяти (8 на процессор) \\

%         \hline
%         Дисковый контроллер & Smart Array P410i RAID контроллер с 1GB  flash backed write cache  (FBWC). Поддержка RAID 0,1 \\
        
%         \hline
%         Поддержка внутренних дисков & 2 отсека для установки накопителей SSD \\

%         \hline
%         Сетевой контроллер & 2 интегрированных двухпортовых контроллера NC551i FlexFabric 10Gb \\

%         \hline
%         Графическая подсистема & ATI Radeon 50, 16MB памяти \\

%         \hline
%     \end{tabular}
% \end{table}
% \fixTableSectionSpace

В качестве модуля доступа к сети был выбран HP 6125G Ethernet Blade Switch~\cite{hp-blade-switch}. Он поддерживает скорость передачи данных до 10~Гбит/с, что обеспечивает высокую производительность для требовательных приложений, таких как виртуализация, обработка больших данных и искусственный интеллект. Стоимость: 752~белорусских рублей~\cite{hp-blade-switch}.

\subsection{Обоснование выбора пользовательского оборудования}

Под рабочим местом следует понимать выделенную часть площади с расположенным на ней технологическим оборудованием, необходимым для выполнения работы. Рабочее место~-- это первичная ячейка производственной структуры предприятия.

\subsubsection{Обоснование выбора пользовательских станций}

Основной задачей при организации рабочих мест была обеспечить комфортные условия работы для сотрудников. В первую очередь рабочие станции должны быть оснащены сетевыми адаптерами с возможностью подключения к LAN и иметь приемлемые параметры для выполнения всех поставленных задач. Одним из критериев, станция должна быть достаточно мощной, чтоб была возможность работы с современными программами. Для этого необходимо минимум 16~Гб оперативной памяти, процессор пятого поколения, наличие современного графического адаптера. Так как было принято решение покупать готовую пользовательскую станцию, он выбирался исходя из требования описанных выше. Была выбрана готовая пользовательская станция Jet Gamer 5i11400FD32HD2SD48X306TIG3W7~\cite{pc}.

Jet Gamer 5i11400FD32HD2SD48X306TIG3W7 обладает следующими характеристиками:

\begin{itemize}
    \item процессор: Intel Core i5 11400F;
    \item ОЗУ: DDR4 32~ГБ;
    \item SSD: 480~Гб;
    \item HDD: 2000~Тб;
    \item графический адаптер: NVIDIA GeForce RTX 3060 Ti;
    \item LAN: 1~Гбит/с;
    \item USB порты: один USB 2.0, один USB 3.2;
    \item стоимость: 1964 белорусских рублей~\cite{pc}.
\end{itemize}

% \begin{table}[ht]
%     \caption{Характеристика компьютера Jet Wizard 5i9400FD}
%     \label{table:func:computer}
%     \begin{tabular}{| >{\raggedright}m{0.397\textwidth}
%                     | >{\raggedright\arraybackslash}m{0.55\textwidth}|}
%         \hline
%         \centering Параметры & \centering\arraybackslash Значение\\

%         \hline
%         Процессор & Intel Core i5 9400F  \\
        
%         \hline
%         ОЗУ & DDR4 24 ГБ \\
        
%         \hline
%         SSD & 240 Гб \\

%         \hline
%         HDD & 2 Тб \\
        
%         \hline
%         Графический адаптер & NVIDIA GTX 1650 \\
        
%         \hline
%         Блок питания & 500 Вт \\

%         \hline
%         LAN  & 1 Гигабит \\
        
%         \hline
%         USB  & 1 USB 2.0, 1 USB 3.2 \\
%         \hline
%     \end{tabular}
% \end{table}
% \fixTableSectionSpace

Также для работы с компьютером необходимо подобрать монитор, клавиатуру и мышь.

Для компьютера было подобрано три схожих монитора: Dell P2419H~\cite{dell_p2419h}, ASUS VA24DQ~\cite{asus_va24dq}, HP EliteDisplay E243~\cite{hp_1fh49aa}. Все они имеют размер 24~дюйма, разрешение Full HD матрицу IPS. HP EliteDisplay E243 и ASUS VA24DQ имеют частоту обновления 75~Гц, а Dell P2419H 60~Гц. Так же монитор от ASUS имеет встроенные колонки, что делает его более многофункциональным.   

Исходя из характеристик был выбран монитор ASUS VA24DQ, потому что обладает наилучшей частотой обновления и имеет встроенные динамики. Стоимость: 496~белорусских рублей~\cite{asus_va24dq}.

Клавиатура для компьютера была выбрана Microsoft Wired Keyboard 600~\cite{wired600usb}. Она обладает прочным корпусом, имеет низкий уровень шума при нажатии клавиш, а также является весьма популярной среди офисных клавиатур. Стоимость: 28~белорусских рублей~\cite{wired600usb}.

В качестве мыши выбрана Dell Optical Mouse MS116~\cite{dell_275bbcb}. Она является полностью симметричной, имеет хороший оптический сенсор, компактный размер и легкий вес, а также длинный провод. Стоимость: 168~белорусских рублей~\cite{dell_275bbcb}.

\subsubsection{Обоснование выбора принтера}

В соответствии с требованиями заказчика необходимо разместить в центре обработки данных несколько принтеров. Учитывая, что работа центра подразумевает большое количество распечатываемых документов, следует выбирать принтеры, которые обеспечивают низкую стоимость и высокую скорость печати. В связи с высокой стоимостью лазерных принтеров, в сети преимущественно используются струйные принтеры. 

Для принтера требования были подходящие порты и форм-фактор. В результате анализа рынка были выбраны струйные принтеры Canon PIXMA G540~\cite{canon_pixmag540}, Epson M1170~\cite{epson_c11ch44404}, HP OfficeJet Pro 8210~\cite{hp_d9l63a} и Xerox B310~\cite{xerox/b310vdni}.

В ходе сопоставления у Canon PIXMA G540 были худшие технические характеристики, по сравнению с HP OfficeJet Pro 8210. В Epson M1170 была более медленная печать и меньший запас прочности за туже цену что у HP OfficeJet Pro 8210. Xerox B310 не поддерживало цветную печать, что являлось исключающим фактором для полного выполнения своих функциональных целей. Исходя из этого было решено выбрать принтер HP OfficeJet Pro 8210.

Основные характеристики струйного принтера HP OfficeJet Pro 8210:
\begin{itemize}
    \item тип: струйный;
    \item формат: A4 (210x297~мм);
    \item ширина бумаги: 356~мм;
    \item печать: цветной;
    \item технология печати: струйный;
    \item количество цветов:~4;
    \item скорость ч/б печати (А4): 22~стр/мин;
    \item скорость цветной печати (А4): 18~стр/мин;
    \item стоимость: 1120 белорусских рублей~\cite{hp_d9l63a}.
\end{itemize}

\subsubsection{Обоснование выбора интерактивной доски}

Одним из важных компонентов для ЦОД является интерактивная доска, которая позволяет визуализировать информацию и проводить презентации.

В процессе анализа рынка для ЦОД выбраны интерактивных доски SMART Board~MX265-V2~\cite{smart_board}, Interwrite MTM-65T9~65"~\cite{interwrite-mtm-65t9}, NextPanel~65P~\cite{interaktivnye_paneli_46152}~-- это все интерактивные доски с диагональю экрана 65 дюймов и разрешением 4K Ultra HD (3840x2160~пикселей). Все они поддерживают мультитач (до 10 точек касания) и имеют встроенное программное обеспечение для совместной работы. 

Основные различия между этими устройствами следующие:
\begin{enumerate_num}
    \item Встроенное программное обеспечение: SMART Board MX265-V2 поставляется с предустановленным программным обеспечением SMART Board, которое считается одним из лучших в отрасли. Interwrite MTM-65T9~65" поставляется с предустановленным программным обеспечением Interwrite ProBoard. NextPanel 65P не поставляется с предустановленным программным обеспечением, но поддерживает работу с различными сторонними приложениями.
    \item Возможность подключения к сети: SMART Board MX265-V2 и Interwrite MTM-65T9 65" поддерживают подключение к сети, что позволяет использовать их в качестве совместной рабочей поверхности. NextPanel~65P не поддерживает подключение к сети.
    \item SMART Board~MX265-V2 устройство имеет встроенный динамик и микрофон, что позволяет воспроизводить звук и записывать аудио без подключения внешних устройств.
\end{enumerate_num}


После изучения данных моделей было решено остановиться на интерактивной доске SMART~Board MX265-V2, потому что остальные модели оказались более низких технических характеристик, но по схожей цене.

Интерактивная доска SMART Board~MX265-V2 имеет следующие характеристики:
\begin{itemize}
    \item диагональ экрана: 65~дюймов;
    \item разрешение экрана: 3840x2160~пикселей;
    \item возможность подключения к сети: Wi-Fi, Ethernet;
    \item поддерживаемые операционные системы: Windows, macOS, Android, iOS;
    \item количество динамиков 2x15~Вт;
    \item стоимость: 19220 белорусских рублей~\cite{smart_board}.
\end{itemize}

\subsection{Обоснование выбора операционной системы}

\subsubsection{Обоснование выбора пользовательской операционной системы}
В качестве операционной системы для пользовательских станций и была выбрана OC Windows~10~\cite{win_10}. Windows 10 является самой популярной настольной операционной системой. Поэтому большинству пользователей будет удобнее и привычнее работать именно с этой операционной системой. Так же эта версия ОС выбрана из-за большого количества поддерживаемого софта в отличии от новой 11 версии.

\subsubsection{Обоснование выбора операционной системы сетевого оборудования}

Используемая сетевая аппаратура производится компанией Mikrotik. В качестве ОС будет использована RouterOS~\cite{router_os}, так как эта операционная система является предустановленной производителем.

\subsubsection{Обоснование выбора операционной системы сервера}

При выборе ОС для блейд-сервера компании HP было обнаруженно что список поддерживаемых ОС ограничен~\cite{server_support_os}. Список поддерживаемых ОС на серверах HPE: Microsoft, VMware, Red Hat, SUSE, Canonical Ubuntu, Oracle Linux and Oracle VM, Citrix, CentOS, ClearOS, SAP Linux. Наиболее популярной ОС для серверов на 2020 год является Windows Server~\cite{server_top_os}. Компания Microsoft предлагает ОС Windows Server~-- серверную операционную систему корпоративного класса с широкими возможностями управления хранением данных, приложениями и сетями. Так же потому что эта ОС является наиболее популярной среди конкурентов большинство софта обязательное будет портировано на нее и как следствие будет иметь постоянную поддержку. Что делает сервера на Windows наиболее универсальными для разных типов задач. Так же имеется возможность задействовать облачные сервера Microsoft Azure~\cite{microsoft_azure} в случае возрастания нагрузки на собственную информационную-систему и использоваться по модели оплаты по мере использования (pay-as-you-go). Исходя из всего вышесказанного на сервера будет установлена Windows Server~2019.

\subsection{Внешняя IPv4 адресация}

Согласно требованиям заказчика, непосредственное подключение к провайдеру отсутствует, то есть сеть соединена с общей сетью здания. 

Согласно варианту, существует выбор из десяти подсетей. Подсети в нотации Classless Inter-Domain Routing (далее~-- CIDR) и количество доступных адресов для конечных устройств отсортированы в порядке возрастания по длине маски:

\begin{enumerate_num}
    \item 157.66.64.0/18~-- 16,382.
    \item 161.19.192.0/18~-- 16,382.
    \item 5.180.32.0/19~-- 8,190.
    \item 126.27.0.0/19~-- 8,190.
    \item 137.14.96.0/20~-- 4,094.
    \item 47.74.44.0/23~-- 510.
    \item 201.113.114.0/24~-- 254.
    \item 199.63.101.128/25~-- 126.
    \item 88.70.8.192/28~-- 14.
    \item 183.255.243.208/28~-- 14.
\end{enumerate_num}

Подсети с седьмой по десятую имеют недостаточное количество хостов, при этом подсети с первой по четвертую имеют значительно избыточное количество хостов. Поэтому для внешнего IP-адреса была использована подсеть 47.74.44.0/23, имеющая 510 адресов, потому что здание также может предоставлять IP-адреса для других компаний, арендующих помещения. Центру обработки данных научно-исследовательского института был выдан адрес 47.74.44.100 и шлюз 47.74.44.99.

\subsection{Внутренняя IPv4 адресация}

Согласно требованиям заказчика, для внутренней IPv4 адресации должны быть использованы публичные адреса. Для ЦОД администратором сети офиса была выделена публичная подсеть 199.63.101.128/25~-- данная подсеть может адресовать до 126 рабочих хостов. Данного количества адресов вполне достаточно, чтобы обеспечить адресацию для всех устройств в локальной сети и при этом оставить запас, для обеспечения возможности расширения сети и добавления новых хостов. 

Как оговаривалось ранее, ЦОД была выделена публичная подсеть 199.63.101.128/25. Прежде чем назначать адреса устройствам сети, необходимо составить схему адресации. Так как в требованиях по безопасности для разрабатываемой сети заказчик не уверен, целесообразно разделение сети на подсети для каждого из VLAN, при этом должно быть учтено различие количества, относящегося к VLAN, хостов. Для более эффективного использования имеющегося адресного пространства, пул адресов будет поделен на множество мелких подсетей используя маски переменной длинны VLSM. Это позволит разграничить адресное пространство под необходимое количество хостов.

Первым шагом нужно определить какие будут выделены подсети и количество устройств в каждой из них. Изучив предприятие было решено выделить беспроводную, стационарную и административную подсети.

Следующим шагом необходимо выполнить разделение на эти подсети. Произведем расчет для беспроводной сети в качестве примера. Исходя из условия что в локальной сети может быть большое скопление людей, сделаем вывод что в помещении может в один момент времени находится около 50 беспроводных подключений. Так же следует учесть что количество устройств может быть больше и необходимо выбрать подсеть с запасом. Расчет будет на 62 устройства в беспроводной сети, что обеспечит 12 резервных адресов. Изначально маска сети 199.63.101.128~-- 25. В бинарном виде маска имеет вид 11111111.11111111.11111111.10000000. Число нулевых битов маски~-- 7, это означает что подсеть может адресовать $2^7-2$ хостов, то есть 126 устройства. Для получения необходимого количества адресов необходимо увеличить число единичных битов маски на один, получим 11111111.11111111.11111111.11000000. Теперь число нулевых битов маски равно шести, а значит подсеть может адресовать $2^6-2$ хостов, то есть 62 устройства. В результате вычислений была получана новая подсеть 199.63.101.128 с маской 16 (число единичных битов). Расчет остальных подсетей будет выполнен аналогичным образом, с увеличением макси для избегания перекрытия и начиная со следующего адреса на котором заканчивается предыдущая подсеть.

Схема IPv4 адресации приведена в таблице~\ref{table:func:ipv4-addresation}.

\begin{table}[ht]
    \caption{Схема внутренней IPv4 адресации}
    \label{table:func:ipv4-addresation}
    \begin{tabular}{|>{\raggedright}m{0.26\textwidth}|>{\raggedright}m{0.22\textwidth}|>{\raggedright}m{0.26\textwidth}|>{\raggedright\arraybackslash}m{0.15\textwidth}|}
        \hline
        Сеть & VLAN & Адрес подсети & Хосты \\
        \hline
        Беспроводная & 10  & 199.63.101.128/26 & 62\\
        \hline
        Стационарная & 20 & 199.63.101.192/27 & 30\\
        \hline
        Административная & 30 & 199.63.101.224/29 & 6\\
        \hline
    \end{tabular}
\end{table}

Беспроводной VLAN 10~-- для беспроводных маршрутизаторов, пользователи которых в разных частях помещения будут видны друг другу посредством статической маршрутизации. 

Стационарный VLAN 20~-- имеет в своем доступе 30 хостов для стационарных подключений с возможностью доступа между собой, а также это реализовано для полноценного доступа к блейд-серверу. 

Административный VLAN 30~-- для адресации, в котором присваиваются адреса активному сетевому оборудованию.

Оборудованию в VLAN 10 назначены следующие статические адреса:
\begin{itemize}
    \item маршрутизатор M1: 199.63.101.129/26;
    \item беспроводной маршрутизатор WR1: 199.63.101.131/26.
\end{itemize}
Остальная адресация беспроводных устройств будет происходить с помощью DHCP из диапазона 199.63.101.133~-- 199.63.101.183

Стационарному оборудованию назначены адреса в VLAN 20:
\begin{itemize}
    \item маршрутизатор M1: 199.63.101.193/27;
    \item персональный компьютер PC2: 199.63.101.197/27;
    \item персональный компьютер PC3: 199.63.101.198/27;
    \item интерактивная доска B1: 199.63.101.199/27;
    \item блейд-сервер S1: 199.63.101.200/27~-- 199.63.101.207/27.
\end{itemize}

Статические адреса для административного VLAN 30:
\begin{itemize}
    \item маршрутизатор M1: 199.63.101.225/29;
    \item персональный компьютер PC1: 199.63.101.227/29;
    \item блейд-сервер S1: 199.63.101.228/29.
\end{itemize}


\subsection{Адресация IPv6}

Согласно требованию заказчика необходимо настроить IPv6 адресацию с доступом в интернет, используя подсеть с блоком адресов для Беларуси~\cite{adresses_rb}. Для этого ЦОД был выдан следующий блок IP-адресов: 2001:67c:2268::/48. Разделение на подсети при помощи уникального идентификатора подсети. Схема адресации IPv6 представлена в таблице~\ref{table:func:ipv6-addresation}.

\begin{table}[ht]
    \caption{Схема IPv6 адресации}
    \label{table:func:ipv6-addresation}
        \begin{tabular}{|>{\raggedright}m{0.37\textwidth}|>{\raggedright}m{0.20\textwidth}|>{\raggedright\arraybackslash}m{0.35\textwidth}|}
        \hline
        \centering Сеть \arraybackslash & \centering VLAN &\arraybackslash \centering Адрес подсети \arraybackslash\\
        \hline
        Беспроводная & 1  & 2001:67c:1058:1::/64\\
        \hline
        Стационарная & 2 & 2001:67c:1058:2::/64\\
        \hline
        Административная & 3 & 2001:67c:1058:3::/64\\
        \hline
    \end{tabular}
\end{table}

Адресация беспроводных устройств будет происходить по DHCPv6, адреса будут выбираться из подсети 2001:67c:1058:1::/64. 

Для оборудования в VLAN 10 были назначены следующие статические IPv6 адреса:
\begin{itemize}
    \item маршрутизатор M1: 2001:67c:1058:1::1/64;
    \item беспроводной маршрутизатор WR1: 2001:67c:1058:1::3/64.
\end{itemize}

Стационарному оборудованию в VLAN 20 были выбраны следующие IPv6 адреса:
\begin{itemize}
    \item маршрутизатор M1: 2001:67c:1058:2::1/64;
    \item персональный компьютер PC2: 2001:67c:1058:2::5/64;
    \item персональный компьютер PC3: 2001:67c:1058:2::6/64;
    \item интерактивная доска B1: 2001:67c:1058:2::7/64;
    \item блейд-сервер S1: 2001:67c:1058:2::8/64~-- 2001:67c:1058:1::15/64.
\end{itemize}

В административном VLAN 30 были выбраны следующие IPv6 адреса:
\begin{itemize}
    \item маршрутизатор M1: 2001:67c:1058:3::1/64;
    \item персональный компьютер PC1: 2001:67c:1058:3::3/64;
    \item блейд-сервер S1: 2001:67c:1058:3::4/64.
\end{itemize}

\subsection{Конфигурация сетевого оборудования}
 
Важным шагом является настройка маршрутизации и обеспечение согласованности конфигурации на всех сетевых устройствах. Это включает в себя настройку маршрутов, обеспечение доступа к сети Интернет, а также управление обменом данными между подсетями. Перед настройкой оборудования необходимо описать термин который присущ только оборудованию MikroTik. Bridge MikroTik~\cite{bridge_mikro}~-- это функция, которая позволяет объединить несколько интерфейсов в одну сеть. Он работает на уровне канального моста в модели OSI. Это означает, что Bridge MikroTik может объединять различные типы интерфейсов, такие как Ethernet, Wi-Fi и другие, в единую сеть. Bridge MikroTik предоставляет широкие возможности для настройки и управления трафиком в сети. Его можно использовать для объединения нескольких сегментов сети, контроля трафика между различными устройствами и сегментами, а также для настройки функций VLAN. Благодаря наличию на выбранном маршрутизаторе CCR2004-16G-2S+PC чипа коммутации в данной ЛКС можно не использовать коммутатор, а использовать только маршрутизатор и технологию Bridge, для коммутации трафика.

\subsubsection{Конфигурация VLAN}

Первым делом создаем Bridge для каждого из VLAN-ов. Указываем имя для сетей:

\begin{lstlisting}
[admin@MikroTik] /interface bridge
[admin@MikroTik] /interface bridge> add name=Wireless
[admin@MikroTik] /interface bridge> add name=Desktop
[admin@MikroTik] /interface bridge> add name=Administrator
\end{lstlisting}

Следующим шагом создаем интерфейс VLAN с идентификаторами 10, 20, 30:

\begin{lstlisting}
[admin@MikroTik] /interface vlan
[admin@MikroTik] /interface vlan> add name=vlan1 vlan-id=10 disabled=no
[admin@MikroTik] /interface vlan> add name=vlan2 vlan-id=20 disabled=no
[admin@MikroTik] /interface vlan> add name=vlan3 vlan-id=30 disabled=no
\end{lstlisting}

Дальше необходимо добавить к Bridge VLAN интерфейсы:

\begin{lstlisting}
[admin@MikroTik]>/interface bridge vlan
[admin@MikroTik]>/interface bridge vlan> add bridge=Wireless vlan-ids=10 interface=ether4
[admin@MikroTik]>/interface bridge vlan> add name=Desktop vlan-ids=20 interface=ether1, ether2, ether7, SFP2 
[admin@MikroTik]>/interface bridge vlan> add bridge=Administrator vlan-ids=30 interface=ether5, ether6
\end{lstlisting}

Следующим шагом необходимо назначить всем созданным выше VLAN-ам диапазон IPv4 и IPv6 адресов выбранных для данной локальной копмьютерной сети:

\begin{lstlisting}
[admin@MikroTik] /ip address
[admin@MikroTik] /ip address> add address=199.63.101.128/26 interface=WirelessVLAN
[admin@MikroTik] /ip address> add address=199.63.101.192/27 interface=DesktopVLAN
[admin@MikroTik] /ip address> add address=199.63.101.224/29 interface=AdministratorVLAN
[admin@MikroTik] /ipv6 address
[admin@MikroTik] /ipv6 address> add address=2001:67c:1058:1::0/64 interface=WirelessVLAN
[admin@MikroTik] /ipv6 address> add address=2001:67c:1058:2::0/64 interface=DesktopVLAN
[admin@MikroTik] /ipv6 address> add address=2001:67c:1058:3::0/64 interface=AdministratorVLAN
\end{lstlisting}

Так же необходимо чтобы запретить маршрутизатору обрабатывать трафик между VLAN 10 и VLAN 20, а так же запретить доступ к административной сети из других подсетей:

\begin{lstlisting}
[admin@MikroTik]>/routing rule
[admin@MikroTik]>/routing rule>add src-address=199.63.101.128/26 dst-address=199.63.101.192/27 action=unreachable
[admin@MikroTik]>/routing rule>add src-address=199.63.101.192/27 dst-address=199.63.101.128/26 action=unreachable
[admin@MikroTik]>/routing rule>add src-address=199.63.101.128/26 dst-address=199.63.101.224/29 action=unreachable
[admin@MikroTik]>/routing rule>add src-address=199.63.101.128/26 dst-address=199.63.101.224/29 action=unreachable
[admin@MikroTik]>/routing rule>add dst−address=199.63.101.224/29 interface=SFP1 action=drop
[admin@MikroTik]>/routing rule>add src-address=199.63.101.224/29 interface=SFP1 action=drop
[admin@MikroTik]>/routing rule>add dst-address=199.63.101.224/29 interface=SFP1 action=drop
[admin@MikroTik]>/routing rule>add src-address=2001:67c:1058:1::/64 dst-address=2001:67c:1058:2::/64 action=unreachable
[admin@MikroTik]>/routing rule>add src-address=2001:67c:1058:2::/64 dst-address=2001:67c:1058:1::/64 action=unreachable
[admin@MikroTik]>/routing rule>add src-address=2001:67c:1058:1::/64 dst-address=2001:67c:1058:3::/64 action=unreachable
[admin@MikroTik]>/routing rule>add src-address=2001:67c:1058:2::/64 dst-address=2001:67c:1058:3::/64 action=unreachable
[admin@MikroTik]>/routing rule>add src-address=2001:67c:1058:3::/64 interface=SFP1 action=drop
[admin@MikroTik]>/routing rule>add dst-address=2001:67c:1058:3::/64 interface=SFP1 action=drop
\end{lstlisting}


% \begin{lstlisting}
% [admin@MikroTik]>/routing rule
% [admin@MikroTik]>/routing rule>add src-address=199.63.101.128/26 dst-address=199.63.101.192/27 action=unreachable
% [admin@MikroTik]>/routing rule>add src-address=199.63.101.192/27 dst-address=199.63.101.128/26 action=unreachable
% [admin@MikroTik]>/routing rule>add src-address=199.63.101.128/26 dst-address=199.63.101.224/29 action=unreachable
% [admin@MikroTik]>/routing rule>add src-address=199.63.101.128/26 dst-address=199.63.101.224/29 action=unreachable
% [admin@MikroTik]>/routing rule>add src-address=199.63.101.224/29 interface=SFP1 action=drop
% [admin@MikroTik]>/routing rule>add dst-address=199.63.101.224/29 interface=SFP1 action=drop
% [admin@MikroTik]>/routing rule>add src-address=2001:67c:1058:1::/64 dst-address=2001:67c:1058:2::/64 action=unreachable
% [admin@MikroTik]>/routing rule>add src-address=2001:67c:1058:2::/64 dst-address=2001:67c:1058:1::/64 action=unreachable
% [admin@MikroTik]>/routing rule>add src-address=2001:67c:1058:1::/64 dst-address=2001:67c:1058:3::/64 action=unreachable
% [admin@MikroTik]>/routing rule>add src-address=2001:67c:1058:2::/64 dst-address=2001:67c:1058:3::/64 action=unreachable
% [admin@MikroTik]>/routing rule>add src-address=2001:67c:1058:3::/64 interface=SFP1 action=drop
% [admin@MikroTik]>/routing rule>add dst-address=2001:67c:1058:3::/64 interface=SFP1 action=drop
% \end{lstlisting}

% Создание списка доступа для блокировки трафика между VLAN 100 и VLAN 200:

% \begin{lstlisting}
%     [admin@MikroTik] /ip firewall address-list
%     [admin@MikroTik] /ip firewall address-list> add list=block-vlan100-vlan200-ipv4 address=199.63.101.128/28
%     [admin@MikroTik] /ip firewall address-list> add list=block-vlan200-vlan100-ipv4 address=199.63.101.143/26
%     [admin@MikroTik] /ip firewall address-list> add list=block-vlan100-vlan200-ipv6 address=2001:67c:1058:1::0/64
%     [admin@MikroTik] /ip firewall address-list> add list=block-vlan200-vlan100-ipv6 address=2001:67c:1058:2::0/64
% \end{lstlisting}

% Создание списка доступа для блокировки доступа к административной сети из других подсетей:

% \begin{lstlisting}
%     [admin@MikroTik] /ip firewall address-list
%     [admin@MikroTik] /ip firewall address-list> add list=block-admin-access-ipv4 address=199.63.101.128/28
%     [admin@MikroTik] /ip firewall address-list> add list=block-admin-access-ipv4 address=199.63.101.143/26
%     [admin@MikroTik] /ip firewall address-list> add list=block-admin-access-ipv6 address=2001:67c:1058:1::0/64
%     [admin@MikroTik] /ip firewall address-list> add list=block-admin-access-ipv6 address=2001:67c:1058:2::0/64
% \end{lstlisting}


% Настройка правил фильтрации трафика:
% \begin{lstlisting}
%     [admin@MikroTik] /ip firewall filter
%     [admin@MikroTik] /ip firewall filter> add chain=forward src-address-list=block-vlan100-vlan200 dst-address-list=block-vlan100-vlan200 action=drop
%     [admin@MikroTik]
%     [admin@MikroTik]
%     [admin@MikroTik]
    
% \end{lstlisting}

\subsubsection{Конфигурация точки беспроводного доступа}

Для начала на точке беспроводного доступа создается Bridge:

\begin{lstlisting}
[admin@MikroTik] /interface bridge
[admin@MikroTik] /interface bridge> add name=bridge1 
\end{lstlisting}

После создания моста нужно добавить интерфейсы в этот мост, так же назначаем на каких портах будет тегированный и нетегированный трафик:

\begin{lstlisting}
[admin@MikroTik] /interface bridge port
[admin@MikroTik] /interface bridge port> add bridge=bridge1 interface=ether1
[admin@MikroTik] /interface bridge vlan
[admin@MikroTik] /interface bridge vlan > add bridge=bridge1 tagged=bridge1 vlan-ids=1
\end{lstlisting}

Для начала настройки беспроводной сети надо создать интерфейс и привязать его к мосту на котором настроены порты для связи с роутером. Создаем три интерфейса: wireless1, wireless2, wireless3:

\begin{lstlisting}
[admin@MikroTik] > /interface bridge port
[admin@MikroTik] /interface bridge port> add bridge-local interface=wireless1 mode=access
[admin@MikroTik] /interface bridge port> add bridge-local interface=wireless2 mode=access
[admin@MikroTik] /interface bridge port> add bridge-local interface=wireless3 mode=access
\end{lstlisting}

После необходимо задать шифрование. Шифрование WPA обеспечивает защиту данных и сети. WPA использует ключ шифрования, называемый предварительно опубликованным ключом PSK для шифрования данных перед их отправкой. Для доступа к этим данным необходимо ввести этот же пароль. Задаем параметры безопасности будущей сети под именем ProfileSecurity. Пароль должен быть не менее восьми символов:

\begin{lstlisting}
[admin@MikroTik] /interface wireless security-profiles
[admin@MikroTik] /interface wireless security-profiles> set [ find default=yes ] supplicant-identity=MikroTik add authentication types=wpa2-psk eap-method="" mode=dynamic-keys name=ProfileSecurity supplicant-identit="" password=aQtL2l4V1
\end{lstlisting}

Последним шагом будет создание трёх точек доступа на разных частотных диапазонах, которые доступны на выбранной точке беспроводного доступа. 

Создание точки доступа на частоте 2.4 ГГц и привязка к интерфейсу wireless1:

\begin{lstlisting}
[admin@MikroTik] /interface wireless1
[admin@MikroTik] /interface wireless> set [ find default-name=wireless1 ] adaptive-noise-immunity=ap-and-client-mode band=2Ghz-b/g/n channel-width=20mMhz country=belarus mode=ap- bridge security-profile=profile_security ssid=first_wifi_room1 wireless-protocol=802.11
\end{lstlisting}

Создание точки доступа на частоте 5 ГГц и привязка к интерфейсу wireless2:

\begin{lstlisting}
[admin@MikroTik] /interface wireless2
[admin@MikroTik] /interface wireless> set [ find default-name=wireless2 ] adaptive-noise-immunity=ap-and-client-mode band=5Ghz-b/g/n channel-width=20mMhz country=belarus mode=ap- bridge security-profile=profile_security ssid=second_wifi_room1 wireless-protocol=802.11
\end{lstlisting}

Создание точки доступа на частоте 5 ГГц и привязка к интерфейсу wireless3:

\begin{lstlisting}
[admin@MikroTik] /interface wireless3
[admin@MikroTik] /interface wireless> set [ find default-name=wireless3 ] adaptive-noise-immunity=ap-and-client-mode band=5Ghz-b/g/n channel-width=40mMhz distance=indoors country=belarus mode=ap- bridge security-profile=profile_security ssid=third_wifi_room1 wireless-protocol=802.11
\end{lstlisting}

\subsubsection{Конфигурация DHCP}

Необходимо настроить DHCP и DHCPv6 для беспроводного VLAN. Для этого необходимо создать диапазоны IPv4 и IPv6 адресов для точки беспроводного доступа и после присовить их DHCP-серверу:
\begin{lstlisting}
[admin@MikroTik] /ip pool
[admin@MikroTik] /ip pool> add name=ipv4-pool-dhcp ranges=199.63.101.133 − 199.63.101.183
[admin@MikroTik] /ipv6 pool
[admin@MikroTik] /ipv6 pool> add name=ipv6-pool-dhcp prefix=2001:67c:1058:1::/64
[admin@MikroTik] /ip dhcp−server
[admin@MikroTik] /ip dhcp−server> add name=WirelessDHCP interface=Wireless address−pool=ipv4-pool-dhcp 
[admin@MikroTik] /ipv6 dhcp−server
[admin@MikroTik] /ipv6 dhcp−server> add name=WirelessDHCPv6 interface=Wireless address−pool=ipv6-pool-dhcp
\end{lstlisting}

\subsubsection{Настройка статической маршрутизации в сети}

Для настройки статической маршрутизации устанавливаем статические IPv4 и IPv6 адреса на маршрутизаторе:
\begin{lstlisting}
[admin@MikroTik] /ip address
[admin@MikroTik] /ip address> add address=47.74.44.100/23 interface=SFP1
[admin@MikroTik] /ip address> add address=199.63.101.129/27 interface=Wireless
[admin@MikroTik] /ip address> add address=199.63.101.193/27 interface=Desktop
[admin@MikroTik] /ip address> add address=199.63.101.194/27 interface=Administrator
[admin@MikroTik] /ipv6 address
[admin@MikroTik] /ipv6 address> add address=2001:67c:1058:1::1000/48 interface=SFP1
[admin@MikroTik] /ipv6 address> add address=2001:67c:1058:1::1/64 interface=Wireless
[admin@MikroTik] /ipv6 address> add address=2001:67c:1058:2::1/64 interface=Desktop
[admin@MikroTik] /ipv6 address> add address=2001:67c:1058:3::1/64 interface=Administrator
\end{lstlisting}

Для маршрутизации на основном маршрутизаторе требуется определить диапазоны IPv4 и IPv6 адресов, предназначенные для беспроводных устройств:

\begin{lstlisting}
[admin@MikroTik] /ip pool
[admin@mikroTik] /ip pool> add name=wireless-router-1 ranges= 199.63.101.133 − 199.63.101.183, 199.63.101.131/26
[admin@MikroTik] /ipv6 pool
[admin@mikroTik] /ipv6 pool> add name=wireless-router-1-v6 prefix=2001:67c:1058:1::/64
\end{lstlisting}

После необходимо прописать статические маршруты на маршрутизаторе:

\begin{lstlisting}
[admin@MikroTik]>/ip route
[admin@MikroTik] /ip route> add src-addres=wireless-router-1 gateway=199.63.101.131/26
[admin@MikroTik] /ip route> add src-address=0.0.0.0/0 dst-address=0.0.0.0/0 gateway=47.74.44.100/23 interface=SFP1
[admin@MikroTik]>/ipv6 route
[admin@MikroTik] /ip route> add src-addres=wireless-router-1-v6 gateway=2001:67c:1058:1::1/64
[admin@MikroTik] /ip route> add src-address=::/0 dst-address=::/0 gateway=2001:67c:1058:1::1000/48 interface=SFP1
\end{lstlisting}

Назначаем статический IPv4 и IPv6 адрес на точке беспроводного доступа:
\begin{lstlisting}
[admin@MikroTik] /ip address
[admin@MikroTik] /ip address> add address=199.63.101.131/26 interface=ether1
[admin@MikroTik] /ipv6 address
[admin@MikroTik] /ipv6 address> add address=2001:67c:1058:1::1/64 interface=ether1
\end{lstlisting}

Маршрутизация на точке беспроводного доступа:
\begin{lstlisting}
[admin@MikroTik]>/ip route
[admin@MikroTik] /ip route> add src-address=0.0.0.0/0 dst-address=0.0.0.0/0 gateway=199.63.101.129/26
[admin@MikroTik]>/ipv6 route
[admin@MikroTik] /ipv6 route> add src-address=::/0 dst-address=::/0 gateway=2001:67c:1058:1::1/64
\end{lstlisting}

\subsection{Настройка блейд-сервера}

Первое, что стоит сделать после покупки и установки сервера HP в стойку~-- это настроить iLO. iLO~-- это механизм управления серверами в условиях отсутствия физического доступа к ним, он позволит администратору настраивать сервер через браузер находясь у себя за рабочим местом.

Первичную настройку нужно проводить через локальную KVM консоль. После включения сервера и загрузки консоли, на экране появится меню с заголовком <<Main Menu>>, с помощью кнопок на консоли необходимо перейти во вкладку <<Enclosure Settings>>, после нажать на <<OA1 IPv4>>, далее перейти в <<OA1 IPv4>>. Дальше необходимо заполнить поля в форме: <<OA1 IP>>: 199.63.101.228, <<Mask>>: 255.255.255.248, <<Gateway>>: 199.63.101.225. После нажать <<Accept>> и после <<OK>>. Далее в нажать на поле <<OA1 IPv6>> и в форме ввести: <<OA1 IPv6>>: 2001:67c:1058:3::5, <<Prefix>>: 64, <<Gateway>>: 2001:67c:1058:3::1. После нажать <<Accept>> и после <<OK>>.

Для повышения безопасности необходимо перейти в <<Main Menu>>, после нажать на <<Enclosure Settings>>, далее перейти в <<Pass>>. В поле ввести четырехзначный пароль. Пароль должен быть полностью случайным и не состоять из очевидных последовательностей.

Следующим шагом необходимо подключиться к блейд-системе с компьютера администратора. Для этого необходимо прописать в поисковой станице веб-браузера адрес сервера, который был задан на предыдущем шаге. На экране появится окно но котором можно увидеть блейд-систему и поле для входа. Данные для входа можно найти на внутреннем лейбле или на наклейке на корпусе сервера, для примера возьмем: логин: <<Administrator>>, пароль: восемь случайных символов, например: Q1W2E3R4. При вводе необходимо соблюдать регистр. 

После в веб-браузере откроется страница HPE BladeSystem Onboard Administrator~\cite{configuration_blade_systeb}, на которой можно посмотреть на блейд-систему, его вид спереди и сзади, количество установленных блейд-серверов, показания датчиков и состояние сервера и многие другие параметры. Дальше, для настройки IPv4 на каждом сервере, в меню слева, необходимо перейти по вкладкам <<Enclosure settings>>, после нажать на <<Enclousure Bay IP Addressing for IPv4>>. Появится таблица в которой нужно галочками в столбце <<Enabled>> необходимо выбрать все восемь лезвий. После необходимо заполнить столбцы: <<EBIPA Address>>, <<Subnet Mask>>, <<Gateway>>. <<EBIPA Address>> для каждого будет поочередно браться из диапазона: 199.63.101.200~-- 199.63.101.207, <<Subnet Mask>> у всех один~-- 255.255.255.224, <<Gateway>> так же у всех одинаковый~-- 199.63.101.196. Если все успешно, то в столбце <<Current Address>> должны появиться адреса назначенные серверам. Следующим шагом необходимо настроить IPv6, переходим в <<Enclosure settings>>, после нажать на <<Enclousure Bay IP Addressing for IPv4>>. Появится таблица, где так же как и в IPv4, необходимо заполнить столбцы <<Enabled>>, <<EBIPA Address>>, <<Gateway>>. В столбце <<Enabled>> галочками выбираем все восемь лезвий, в <<EBIPA Address>> указываем адреса, обязательно с префиксом, из диапазона: 2001:67c:1058:2::8~-- 2001:67c:1058:2::15. В столбце <<Gateway>> для всех выставляем шлюз 2001:67c:1058:2::1. Если все успешно то в столбце <<Current Address>> появятся адреса назначенные серверам.

Для установки ОС на блейд-сервера необходимо в левом меню раскрыть пункт <<Enclosure settings>>, после нажать на <<Information Device Bays>>, где будут названия всех установленных блейд-серверов. После необходимо выбрать первое лезвие и нажать на него. На экране появится страница с заголовком <<Device Bay Informaition~-- Proliant BL685c G7 (Bay 1)>>. Ниже на этой странице нужно выбрать пункт <<iLO>> и в появившемся меню нажать <<Web Administrator>>. Появится новая страница, где мы должны увидеть пункт <<Last Used Remote Console>> и нажать возле этого пункта <<Launch>>, после этого должна запуститься консоль блейд-сервера в веб-браузере. Далее необходимо навести мышью на верхнюю границу этой консоли, где появится интерактивная панель в которой нужно нажать на значок диска и в появившемся меню, напротив надписи <<Image>>, нажать кнопку <<Mount>>. После этого консоль перезагрузится и после некоторой загрузки в консоли появится привычная форма для установки Windows Server 2019, где необходимо перейти по всем полям заполняя их необходимыми настройками и после начнется установка системы. После установки и автоматический перезагрузки необходимо установить пароль для входа в систему. Для надежности возьмем пароль: x7apmk1YAswGJXp. После этого введем этот пароль в поле входа и попадем в систему, где в последующем можно установить все необходимые программы которые будут использоваться на сервере.

После выходим из ОС и повторяем данные действия для каждого лезвия в пункте <<Device Bays>>.

\subsection{Настройка персональных станций}

Для обеспечения функционирования в ЛКС пользовательских станций необходимо настроить статическую маршрутизацию IPv4 и IPv6. Процесс настройки IPv4 адресов на компьютерах под управлением операционной системы Windows выполняется в соответствии с следующим алгоритмом:
\begin{enumerate_num}
    \item Зайдите в <<Панель управления>>, дальше в <<Сеть и Интернет>>, после в <<Центр управления сетями и общим доступом>>.
    \item Нажмите <<Изменение параметров адаптера>>.
    \item Выберите <<Ethernet>>.
    \item Выберите <<IP версии 4 (TCP/IP)>>, нажмите кнопку <<Свойства>>. 
    \item Выберите <<Использовать сделающий IP-адресс>>. И введите IP-адрес 199.63.101.197, маска подсети 255.255.255.224, основной шлюз 199.63.101.193.
    \item Нажмите <<ОК>>.
\end{enumerate_num}

Алгоритм настройки IPv6:
\begin{enumerate_num}
    \item Зайдите в <<Панель управления>>, дальше в <<Сеть и Интернет>>, после в <<Центр управления сетями и общим доступом>>.
    \item Нажмите <<Изменение параметров адаптера>>.
    \item Выберите <<Ethernet>>.
    \item Выберите <<IP версии 6 (TCP/IP)>>, нажимаем кнопку <<Свойства>>. 
    \item Выберите <<Использовать сделающий IP-адресс>>. И заполните поля: IPv6-адрес 2001:67c:1058:2::5, длина префикса подсети 64, основной шлюз 2001:67c:1058:2::1
    \item Нажмите <<ОК>>.
\end{enumerate_num}

Для второй пользовательской станции действия буду аналогичны, отличаться будет только данные в форме ввода. Для IPv4 нужно будет ввести: IP-адрес 199.63.101.198, маска подсети 255.255.255.224, основной шлюз 199.63.101.193. А для IPv6: IPv6-адрес 2001:67c:1058:2::6, длина префикса подсети 64, основной шлюз 2001:67c:1058:2::1.

На станции администратора для IPv4 нужно будет ввести: IP-адрес 199.63.101.225, маска подсети 255.255.255.248, основной шлюз 199.63.101.225. А для IPv6: IPv6-адрес 2001:67c:1058:3::1, длина префикса подсети 64, основной шлюз 2001:67c:1058:3::1.

\subsection{Настройка интерактивной доски}

Для подключения интерактивной доски к ЛКС необходимо выполнить следующие шаги:
% \begin{enumerate_num}
%     \item Подключите интерактивную доску SMART Board MX265-V2 к компьютеру при помощи USB-кабеля. Удостоверьтесь, что кабель подключен к компьютеру и доске надежно.
%     \item Посетите официальный веб-сайт SMART Technologies[?] и загрузите последние драйверы.
%     \item Установите драйверы, следуя инструкциям установщика.
%     \item Откройте программное обеспечение SMART Board на вашем компьютере. Которое установится в процессе загрузки драйверов.
%     \item Проверьте функциональность, а также убедитесь, что программное обеспечение корректно распознало доску.
% \end{enumerate_num}

\begin{enumerate_num}
    \item Подключите кабель Ethernet к разъему RJ45 на дисплее интерактивной доски.
    \item Включите интерактивную доску с помощью устройства управления.
    \item На экране выберите <<Settings>>.
    \item Нажмите на пункт меню <<Network>>.
    \item Убедитесь, что Ethernet включен, а Wi-Fi отключен.
    \item В меню выберите <<IPv4>>.
    \item Открывается окно настройки IP и DNS.
    \item Введите IP адрес 199.63.101.199, шлюз по умолчанию 199.63.101.193 и маску подсети 255.255.255.224.
    \item Нажмите <<Save>>.
    \item В меню выберите <<IPv6>>.
    \item Введите IP адрес 2001:67c:1058:2::7, шлюз по умолчанию 2001:67c:1058:2::1 и префикс подсети 64.
    \item Нажмите <<Save>>.
\end{enumerate_num}

Для возможности подключения компьютера к интерактивной доске есть несколько способов, по HDMI кабелю, по USB кабелю, через локальную сеть. В случае подключения через HDMI нужно просто соединить кабелем доску и компьютер, выбрать в настройках интерактивной доски HDMI порт, и экран компьютера будет транслироваться на доску. В случае подключения через сеть или ЛКС на  необходимо выполнить шаги:
Следующим шагом будет настройка на компьютере :
\begin{enumerate_num}
    \item Подключите интерактивную доску SMART Board~MX265-V2 к компьютеру при помощи USB-кабеля или подключитесь к той же сети что и доска. Удостоверьтесь, что кабель подключен к компьютеру и доске надежно.
    \item Посетите официальный веб-сайт SMART Mirror~\cite{smart_mirror} и загрузите приложение на компьютер.
    \item Установите приложение следуя указаниям разработчика.
    \item Запустите приложение.
    \item На интерактивной доске, снизу в меню, нажмите <<Screen Share>>. На экране должен появиться шестизначный код.
    \item Введите данный код в приложении на компьютере.
\end{enumerate_num}

\subsection{Настройка принтера}

Принтер будет подключаться к компьютеру по USB. Для успешной работы необходимо сделать следующим инструкциям:
\begin{enumerate_num}
    \item Подсоедините USB-кабель принтера к компьютеру.
    \item Если отображается сообщение <<Найдено новое оборудование>>, следуйте запросам на экране для установке драйвера, затем отправьте задание печати, чтобы проверить подключение. Если подключение принтера или задание печати завершается со сбоем, выполните следующие действия.
    \item Найдите и откройте пункт <<Принтеры и сканеры>>.
    \item Нажмите <<Добавить принтер или сканер>>.
    \item Если принтер присутствует в списке, выберите его, щелкните <<Добавить устройство>>, затем следуйте запросам для установки драйвера.
    \item Если ваш принтер не указан в списке, перейдите к выполнению следующих действий.
    \item Нажмите <<Необходимый принтер>> отсутствует в списке, а затем выберите <<Добавить локальный принтер или сетевой принтер с ручными параметрами>>.
    \item Нажмите <<Далее>>, выберите Использовать существующий порт, выберите <<USB001: виртуальный порт принтера для USB>> в раскрывающемся меню, затем нажмите кнопку <<Далее>>,.
    \item По запросу выберите <<Центр обновления Windows>>, затем дождитесь завершения обновления драйверов печати.
    \item В разделе <<Производитель>> выберите HP или Hewlett Packard, затем выберите название вашего принтера.
    \item Нажмите <<Далее>>, затем следуйте запросам для установки драйвера.
    \item Отсоедините, затем снова подсоедините кабель USB, чтобы завершить настройку принтера.
\end{enumerate_num}

\subsection{Настройка возможности просмотра видео посредством беспроводной сети}

Для возможности просмотра видео посредством беспроводной сети был выбран протокол DLNA~-- набор стандартов, позволяющих совместимым устройствам передавать и принимать по домашней сети различный медиаконтент, а также отображать его в режиме реального времени.

В Windows 10 можно использовать функции DLNA для воспроизведения контента, не настраивая DLNA-сервер. Единственное требование~-- чтобы и компьютер и устройство, на котором планируется воспроизведение были в одной локальной сети.

В нашей сети обеспечивается возможность передачи данных за счет возможности подключение компьютеров к беспроводной сети.

Первым действием в меню пуск открываем <<Параметры потоковой передачи мультимедиа>>. Следующим действием нажимаем «Включить потоковую передачу мультимедиа». Далее вводим имя DLNA-сервера, например VLAN 10. Дальше выбрав устройство и нажав «Настроить» можно указать, к каким типам медиа следует предоставлять доступ. 

Для приема на Android устройство необходимо установить приложение DLNA server. После запускаем его выбираем необходимое нам устройство и получаем доступ к папке с возможностью потоковой передачи видео. 

\subsection{Защита от вирусов}

Современные корпоративные сети требуют высочайшего качества защиты: всего один вредоносный файл может быстро инфицировать все узлы сети, нарушить бизнес-процессы и вывести из строя IT-инфраструктуру компании. Защищать сервера и компьютеры организации должны специализированные решения способные надежно защитить критически важные данные от новейших вредоносных программ, обеспечивая бесперебойную работу в условиях самой высокой нагрузки при минимальном потреблении ресурсов системы. 

Было принято решение установить антивирусное ПО на серверное оборудование и на компьютеры пользователей. Так как у нас на всех устройствах установлена операционная система Windows можно рассматривать антивирус от одного производителя на все устройства. После анализа рынка были выбраны три наиболее популярных антивируса: Dr.Web~\cite{drweb}, Kaspersky~\cite{kaspersky}, ESET NOD32~\cite{esetnod32}. Все они имеют возможность рабоать на ОС Windows Server, что было ключевым фактором при выборе. Вот некоторые особенности этих антивирусов:
\begin{enumerate_num}
    \item Dr.Web известен своей высокой эффективностью в обнаружении и удалении сложных вредоносных программ, таких как вирусы-вымогатели и шпионские программы. Он также предлагает ряд дополнительных функций, таких как защита от фишинга и вредоносных ссылок в электронной почте и веб-браузере.
    \item Kaspersky также обеспечивает надежную защиту от широкого спектра угроз. Он отличается простым и интуитивно понятным интерфейсом, а также широким набором функций, включая родительский контроль и защиту от кражи данных.
    \item ESET NOD32 отличается высокой скоростью сканирования и минимальной нагрузкой на систему. Он также предлагает ряд дополнительных функций, таких как защита от вредоносных программ в режиме реального времени и защита от кражи паролей.
\end{enumerate_num}

Так как ЦОД нужен сервис с самой эффективностью в обнаружении сложных вредоносных программ, то Dr.Web~-- лучший выбор. Благодаря тому что в ЦОД выбрано достаточно производительное оборудование не будет проблем с тем чтоб использовать этот антивирус и не тормозить систему.

Для начала работы с антивирусом необходимо купить лицензию на официальном сайте~\cite{products_drweb}. Есть различные решения для рабочих станция и для серверов. Нам необходимо приобрести лицензию на три компьютера и восемь серверов. Итоговая стоимость за один год использования будет 663~белорусских рубля~\cite{products_drweb}.

Установка на устройства не будет существенно отличаться, поэтому будет описать процесс установки на один блейд-сервер:
\begin{enumerate_num}
    \item Для установки необходимо скачать ПО c официального сайта.
    \item С помощью компьютера администратора зайти в ОС на блейд-сервере.
    \item Смонтировать носитель с загруженным антивирусным ПО.
    \item Запустить установочное ПО и следовать всем инструкциям мастера установки.
    \item На этапе запроса ключа, необходимо ввести ключ который был отправлен на почту после покупки.
\end{enumerate_num}

Далее этапы установки необходимо повторить на каждом устройстве.
