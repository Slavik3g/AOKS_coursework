\sectionCenteredToc{ЗАКЛЮЧЕНИЕ}
\label{sec:outro}

В ходе выполнения данного курсового проекта была разработана и спроектирована локальная компьютерная сеть для центра обработки данных научно-исследовательского института. Проект включал в себя как теоретические, так и практические аспекты проектирования локальных компьютерных сетей. Была разработана структурная кабельная система, произведен подбор оборудования и расходных материалов, подготовлена настройка активного сетевого оборудования и оконечных устройств.  

Одним из ключевых этапов проектирования стало создание структурной кабельной системы, которая была разработана с учетом особенностей здания. После тщательного подбора оборудования и расходных материалов была осуществлена настройка активного сетевого оборудования и оконечных устройств.

В результате были созданы структурные и функциональные схемы, план здания, а также перечень необходимого оборудования и материалов для успешной реализации сетевого проекта. В этот перечень вошли различные элементы, такие как маршрутизаторы, рабочие станции, принтеры, блейд-сервер, интерактивная доска, точки доступа и пассивное сетевое оборудование. Следует отметить, что все выбранные компоненты соответствуют высоким стандартам качества и надежности.

Кроме того, важно отметить, что при разработке данной локальной компьютерной сети был уделен особый акцент на обеспечение безопасности. На все устройства были установлены современные антивирусы. А также отдельное внимание было уделено возможности беспроводной сети функционировать при большом количестве пользователей.

Разработанная компьютерная сеть обладает не только высокой функциональностью, но и проста в обслуживании. Ее архитектура также готова к масштабированию в случае необходимости.