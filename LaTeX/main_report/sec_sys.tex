\section{СТРУКТУРНОЕ ПРОЕКТИРОВАНИЕ}
\label{sec:sys}

В данном разделе описывается проектирование структурной схемы ЛКС. Структурная схема представлена в приложении A.

\subsection{Блок интернета}

Блок интернета служит для обеспечения доступа ЦОД к внешним ресурсам, облачным сервисам и для обмена данными с внешними устройствами и системами.

Данный блок связан с маршрутизатором. Это обосновано тем, что маршрутизатор выполняет функцию пересылки трафика между внутренней сетью и интернетом. Связь с другими блоками, такими как блок коммутации, блок оконечных устройств, блейд-сервера и блоком беспроводных точек доступа обеспечивается через блок маршрутизации, так как он выполняет роль шлюза между внутренней и внешней сетями.

\subsection{Блок маршрутизации}

Маршрутизатор используется для обеспечения маршрутизации данных между внутренней локальной сетью и внешней сетью (интернетом). Он выполняет функции маршрутизации, фильтрации трафика и обеспечивает безопасность сети.

Маршрутизатор связан с блоком интернета для доступа к внешней сети. Он также связан с блоком коммутации, так как обеспечивает маршрутизацию трафика между различными устройствами внутри локальной сети.

\subsection{Блок коммутации}

Блок коммутации реализован на основе маршрутизатора. Это решение принято по причине того, что количество стационарных устройств не большое и соответственно нет необходимости усложнять сетевую инфраструктуру.

Данный блок используются для обеспечения связности между оконечными устройствами и другими устройствами внутри сети. Они обеспечивают высокую пропускную способность и низкую задержку внутри локальной сети. Данный блок соединен с блоком маршрутизации, чтобы обеспечить доступ к внешней сети, включая Интернет, и маршрутизацию данных между внутренней сетью и внешними ресурсами. А так же блок связан с блоком оконечных устройств для обеспечения связности и обмена данными между компьютерами, а также с блейд-сервером и блоком беспроводных точек доступа.

\subsection{Блок оконечных устройств}

Блок оконечных устройств включает в себя компьютеры, принтеры и интерактивные доски используемые сотрудниками компании.

Оконечные устройства используются сотрудниками для работы с данными, доступа к серверу, выполнения задач и коммуникации внутри локальной сети.

Оконечные устройства не связаны между собой напрямую, а взаимодействуют с друг другом благодаря подключения к блоку коммутации. Это позволяет сотрудникам передавать данные и обмениваться информацией внутри компании, а также иметь доступ к сетевым ресурсам, таким как сервер и интерактивная доска.

\subsection{Блок блейд-сервера}

Блок блейд-сервера является важной частью инфраструктуры ЦОД в научно-исследовательском институте. Блейд-сервер представляет собой высокопроизводительный компьютерный сервер, которые обеспечивают обработку данных, хранение информации и выполнение вычислительных задач. 

Блок блейд-сервера подключен к блоку коммутации, это позволяет обмениваться данными и управлять блейд-сервером устройствам, находящимся в локальной сети.

\subsection{Блок точек доступа}

Согласно требованию заказчика, должно быть предусмотрено подключение беспроводных устройств. Для этого был выделен блок точек доступа, он будет обеспечивать подключение беспроводных устройств, таких как смартфоны пользователей, и позволять им передавать между друг другом данные, а так же выдавать доступ в интернет.

\subsection{Блок беспроводных устройств}

Блок беспроводных устройств включает в себя беспроводные клиентские устройства, такие как ноутбуки, смартфоны и планшеты, которые используются сотрудниками внутри ЦОД. Эти устройства получают подключение к локальной сети через беспроводные точки доступа. Блок беспроводных устройств упрощает мобильность и гибкость работы внутри ЦОД.

