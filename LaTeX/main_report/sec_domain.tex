\section{ОБЗОР ЛИТЕРАТУРЫ}
\label{sec:domain}

Для выполнения курсового проекта использовались знания, полученные в ходе дисциплин «Теоретические основы компьютерных сетей», «Администрирование компьютерных систем и сетей» и «Аппаратное обеспечение компьютерных сетей». А также использовалась учебная и научная литература~\cite{book_komp_science_olifer, book_komp_science_tanen} и различные электронные ресурсы: статьи, документы и материалы производителей сетевого оборудования.

\subsection{Оптоволокно OS1}

При разработке компьютерных сетей важно уделять внимание не только высоким техническим характеристикам оборудования, но также исследовать среду передачи данных, к которой это оборудование подключается.

Оптический волоконный кабель, также известный как волоконно-оптический кабель, представляет собой совершенно другой тип передающего сигнала в сравнении с традиционными медными кабелями. Его уникальность заключается в передаче информации с использованием световых волн. Основной строительный блок~-- это прозрачное стеклянное волокно, по которому световой сигнал проходит на значительные расстояния, достигая порой нескольких десятков километров, с минимальными потерями.

Оптический волоконный кабель обладает выдающимися характеристиками в сфере защиты от помех и конфиденциальности передаваемой информации. Внешние электромагнитные воздействия не влияют на световой сигнал, и сам сигнал не генерирует внешних электромагнитных излучений. Попытки несанкционированного доступа к данному типу кабеля для прослушивания сети фактически невозможны, так как это потребовало бы нарушения целостности самого кабеля.

Однако оптоволоконный кабель имеет и некоторые недостатки. Самый главный из них~-- высокая сложность монтажа (при установке разъемов необходима микронная точность, от точности скола стекловолокна и степени его полировки сильно зависит затухание в разъеме). Применяют оптоволоконный кабель только в сетях с топологией «звезда» и «кольцо».

Зачастую обеспечение доступа здания к сети интернет создают, используют оптоволокно, а внутри помещения уже используется другой тип соединения для коммутации устройств внутри сети. Для перехода оптоволокна к среде передачи данных, используемой внутри здания, используются различные устройства-адаптеры. 

Существуют многомодовые и одномодовые оптический кабели. Одномодовое волокно разделяется на OS1 и OS2, которые соответствуют спецификациям кабеля SMF~\cite{optovolokno_what_is}. В большинстве случаев волокна OS1 это внутренние кабели, предназначенные для передачи данных на небольшие расстояния, например, в кампусных сетях или в ЦОДах. А волокна OS2, как правило, используются в магистральных внешних кабелях, предназначенных для передачи данных на большие расстояния, с защитой и прямым захоронением. Внутренние кабели OS1 имеют потери на 1~км больше, чем внешние OS2. Максимальное затухание для кабелей OS1 составляет 1~дБ/км, а для OS2~-- 0.4~дБ/км. В результате, максимальное расстояние передачи для кабелей стандарта OS1 составляет 2~км, а для кабелей OS2 может достигать от 5 до 10~км. Поэтому кабели стандарта OS1 намного дешевле, чем кабели OS2. Следует обратить внимание, что оба типа кабелей ~-- и OS1, и OS2 ~-- поддерживают скорость передачи данных 1 и 10~Гбит/с на соответствующие расстояния (2 и 5-10~км). Одномодовые оптические кабеля по стандарту имеет желтую маркировку~\cite{optovolokno_color}.

\subsection{Блейд-сервер}

На момент написания курсового проекта существует большой класс задач, требующих высокой концентрации вычислительных средств. К ним могут относиться как сложные ресурсоемкие вычисления (научные задачи, математическое моделирование, вычислительный поиск), так и обслуживание большого числа пользователей (распределенные базы данных, Интернет-сервисы и хостинг, серверы приложений). Мощность вычислительного центра можно сделать больше, увеличив производительность отдельных вычислительных модулей или их количество. В настоящее время преобладает вторая тенденция, и усилия разработчиков направлены, прежде всего, на внедрение параллельных вычислений. Увеличение числа вычислительных модулей в вычислительном центре требует новых подходов к размещению серверов. Применение кластерных решений приводит к росту затрат на помещения для центров обработки данных, их охлаждение и обслуживание. Решить некоторые из этих проблем поможет новый тип серверов~-- модульные, чаще называемые блейд-серверами, или серверами-лезвиями.

По определению IDC~\cite{idc}, блейд-сервер представляет собой модульную  компьютерную систему, включающую процессор и память. Лезвия вставляются в специальное шасси с объединительной панелью, обеспечивающей им подключение к сети и подачу электропитания. Это шасси с лезвиями, по мнению IDC, является блейд-системой. Оно выполнено в конструктиве для установки в стандартную 19-дюймовую стойку и в зависимости от модели и производителя, занимает в ней 3U, 6U или 10U (один U~-- unit, или монтажная единица, равен 1,75~дюйма). За счет общего использования таких компонентов, как источники питания, сетевые карты и жесткие диски, блейд-сервера обеспечивают более высокую плотность размещения вычислительной мощности в стойке по сравнению с обычными тонкими серверами высотой 1U и 2U. Фактически блейд-система состоит из следующих компонентов:
\begin{enumerate_num}
    \item Блейд-сервера (фактически это обычные сервера без блока питания, с пассивными радиаторами и без PCI разъемов).
    \item Корпус и пассивная плата обеспечивающая коммутацию установленного оборудования.
    \item Системы питания и охлаждения.
    \item Внешние коммутационные устройства (Ethernet, FC, Infiniband).
\end{enumerate_num}

Блейд-сервера являются крайне эффективным решением для экономии пространства в ЦОД, а также с точки зрения их консолидации и перехода к централизованному управлению серверным парком. Например, системный администратор может управлять шасси с лезвиями как одним объектом и по мере роста нагрузок увеличивать его вычислительную мощность, добавляя новые лезвия. Кроме того, поскольку обычно в шасси предусмотрена возможность установки сетевых коммутаторов, эта опция позволяет провести и консолидацию сетевых ресурсов ЦОД.

\subsection{Работоспособность беспроводной сети при скоплении людей}

Сети Wi-Fi несомненно занимают одно из важнейших мест среди технологий радиодоступа, однако, на момент написания курсового проекта, все к большему числу беспроводных сетей предъявляются требования поддержки высокой плотности пользователей. Под сетями Wi-Fi высокой плотности понимается беспроводная среда с высокой концентрацией пользователей, где пользователи подключены к беспроводной сети и интенсивно работают с сетевыми сервисами. В большинстве случаев за пороговое значение принимается одно устройство на квадратный метр~\cite{wi-fi_vysokoy_plotnosti}. 

В связи с тем, что все больше устройств с поддержкой последних версий стандарта IEEE 802.11~\cite{ieee_802.11} выходит на рынок, количество клиентских устройств в беспроводных сетях увеличивается. В случае таких объектов как стадионы или выставочные залы, высокая клиентская плотность этих объектов может внести существенное влияние на работу сети и потребовать применения специфических проектных решений. Основными факторами, влияющими на функционирование сетей Wi-Fi высокой плотности, являются: скорость передачи данных, количество и плотность точек доступа, количество и плотность клиентов, характеристики среды, аппаратные возможности точек доступа и фактические возможности по монтажу.

Большинство WLAN~\cite{what_is_wlan} проектируется для развертывания в офисах, складах, гостиницах и так далее. На подобных площадках обычно требуется беспроводная сеть, спроектированная исходя из среднего числа пользователей с относительно большой зоной покрытия каждой точки доступа. Площадки с высокой плотностью пользователей имеют другие требования, делающие эту модель неприменимой. То, что работает в офисе, не будет работать на стадионе или в помещении с большим количеством людей. 

Важнейшей особенностью сетей высокой плотности является необходимость обеспечивать работу большого числа пользователей, расположенных близко друг к другу. Большое число пользователей и устройств требует большого числа точек доступа, что в свою очередь ведет к высокой интерференции. Интерференция~\cite{interfer} – важнейший движущий фактор при проектировании беспроводной сети. В инженерной практике существует несколько подходов, комбинирование которых позволяет минимизировать данный эффект: 
\begin{itemize}
    \item расширение используемого спектра в доступных диапазонах (больше каналов~-- больше потенциальная пользовательская емкость);
    \item увеличение числа точек доступа, но не более чем необходимо для достижений целевой емкости (больше~-- не значит лучше);
    \item повторное использование частот;
    \item использование направленных антенн.
\end{itemize}

\subsection{Возможность просмотра видео посредством беспроводной сети}

% Согласно документации YouTube скорость необходимая для просмотра видео может варьироваться в зависимости от разрешения. В таблице~\ref{table:func:video_speed} приведена информация какая скорость нужна для воспроизведения видео в том или ином разрешении.

% \begin{table}[ht]
%     \caption{Скорость необходимая для воспроизведения видео в том или ином разрешении}
%     \label{table:func:video_speed}
%     \begin{tabular}{| >{\raggedright}m{0.397\textwidth}
%                     | >{\raggedright\arraybackslash}m{0.55\textwidth}|}
%         \hline
%         \centering Разрешение видео & \centering\arraybackslash Оптимальная постоянная скорость\\

%         \hline
%         4K UHD & 20 Мбит/с \\
        
%         \hline
%         HD – 1080p & 5 Мбит/c \\
        
%         \hline
%         HD – 720p & 2,5 Мбит/c \\
        
%         \hline
%         SD – 480p & 1,1 Мбит/с \\
        
%         \hline
%         SD – 360p & 0,7 Мбит/с \\
        
%         \hline
%     \end{tabular}
% \end{table}

Для возможности просмотра видео посредством беспроводной сети есть несколько способов. 

Первый и наиболее доступный способ на момент написания курсового проекта~-- это DLNA~\cite{dlna_what_is}. Это самая простая и самая распространенная на сегодняшний день технология передачи фильмов и музыки по Wi-Fi. Сам принцип DLNA заключается в том, что на компьютере запускается сервер, в котором прописана открытая для просмотра папка с кино и музыкой. Устройство которое хочет получить данные подключается по Wi-Fi к общей папке и выводит из нее данные. DLNA так же можно настроить на передачу рабочего стола, но технология не рассчитана на это и высокой производительности ожидать не стоит. 

Второй способ~-- беспроводные серверы презентаций. Идея представляет собой следующее: к точке доступа, помимо интернета, можно подключить телевизор или монитор с колонками, и с помощью небольшой утилиты выводить картинку и звук с персонального компьютера (ПК) на телевизор/монитор/колонки. Это и есть схема работы беспроводного сервера презентации. Грубо говоря, это точка доступа, который помимо интернета предоставляет вашему ПК/смартфону устройство вывода изображения и звука.

В большинстве случаев, когда вы хотите передать через сервер презентаций уже не рабочий стол и программы, а кино и музыку, в передающую утилиту встроен плеер, открывая которым ваши медиафайлы они передаются на точку доступа уже по протоколу DLNA, то есть в оконном режиме вы кино не посмотрите, только на полный экран. Когда фильм/музыка заканчивается, приложение автоматически переключается обратно, в режим передачи изображения рабочего стола.

В разное время подобные точки доступа выпускали D-Link, Planet, Edimax, ViewSonic и другие, менее известные авторы. Но наилучшего результата по качеству передачи и снижению задержки достигла фирма Awind со своим продуктом McTivia~\cite{awind-mctivia-review}: это небольшая точка доступа поддерживающая cтандарт  802.11n и имеющая HDMI выход. Также есть вход Ethernet для предоставления общего доступа к сети, а также имеется USB-вход для клавиатуры/мыши, чтобы можно было управлять компьютером, на котором запущена утилита, удаленно.

Так же необходимо выделить общие достоинства и недостатки, обусловленные использованием Wi-Fi для возможности просмотра видео посредством беспроводной.

Достоинства:
\begin{itemize}
    \item большинство компьютеров уже оснащено Wi-Fi адаптером~-- не нужен отдельный передатчик, все, что нужно для трансляции уже имеется;
    \item можно использовать не только для беспроводной передачи видео и звука но и для получения доступа к сети;
    \item благодаря широкой распространенности обращает на себя внимание крупнейших участников IT-индустрии, таких как Intel, Apple, Qualcomm, Cavium Networks.
\end{itemize}

Недостатки:
\begin{itemize}
    \item беспроводная передача видео и звука отнимает часть эфира у прямого назначения Wi-Fi~-- доступа в сеть;
    \item для того, чтобы HD-видео и звук помещались в полосу пропускания Wi-Fi, требуется «упаковать» их соответствующим кодеком (в большинстве случаев~-- H.~264), что дает потерю качества;
    \item потребность в сжатии рождает потребность в софте, который может работать на одной, но не работать на другой операционной системе (ОС);
    \item работа кодека по сжатию контента требует аппаратных ресурсов, при том немалых.
\end{itemize}

\subsection{Компания MikroTik}

MikroTik~\cite{mikrotik}~-- латвийская компания, занимающая важное положение в мире сетевых технологий. Она была основана в 1996 году Андрисом Рагозиншем и с тех пор стала признанным лидером в производстве сетевого оборудования. Основой их успеха является операционная система RouterOS~\cite{router_os}, предоставляющая высокую гибкость и функциональность в настройке и управлении сетевыми устройствами.

Одной из выдающихся черт MikroTik является разнообразие их продуктового портфеля. Они предлагают маршрутизаторы, коммутаторы, беспроводные точки доступа, антенны и другие устройства, обеспечивая полный спектр решений для различных потребностей пользователей. Многие из их устройств, известных как RouterBOARD~\cite{routerboard}, представляют собой компактные микрокомпьютеры с предустановленной RouterOS, что делает их популярным выбором для создания собственных маршрутизаторов и сетей.

Компания также активно участвует в образовании через программу MikroTik Academy~\cite{mikrotik_academy}, предоставляя материалы и ресурсы для студентов и профессионалов в области сетевых технологий. Они организуют MikroTik User Meetings~\cite{mum_mikrotik}, где обмен опытом, обучение и обсуждение новых тенденций становятся ключевыми моментами.

MikroTik также известна своим присутствием в промышленных секторах, где их оборудование используется в проектах, связанных с управлением светофорами, мониторингом энергосистем и другими приложениями.




